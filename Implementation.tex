\section{Implementation}
Sketch\&Stitch's pipeline begins after the user calibrates her fabric and marker colors. users capture an image of their design on fabric. The design is first localized in the image, processed to reduce noise and enhance color, split into design layers based on a color scheme, fiducial Circuit Stickers are detected and replaced with appropriate patterns, design layers are sent to embroidery software to be digitized, and finally, digital embroidery patterns are sent to embroidery machine to be stitched. 


\subsection{Design localization and noise reduction}
Once an image is captured, the system looks for visual markers on the hoop to calculate its center and orientation, and applies the appropriate cropping mask to extract the design and store it in life-size.

The cropped image is then processed: adjust contrast, reduce colors that have been introduced during capturing, and merge redundant colors, e.g., caused by change in amount of pressure applied on markers, and remove background (a solid color). 


%The quality of embroidery depends highly on the quality of the captured image. Factors that affect the quality of an image include camera resolution, light setup and shadows, the flatness and texture of fabric, the quality of markers, and the amount of presser applied to makers while sketching.



\subsection{Color filtering and marker detection}
Next, the system performs color filtering based on a predefined color scheme (HSV model) to split the design into multiple layers. In our implementation we used OpenCV libraries for image processing and marker recognition. 
%ArUco

First, we filter for Circuit Stickers' colors (black and white). Our stickers only use one fiducial marker to signify a circuit bridge. The system looks for fiducial markers and replaces them with the bridge's three layer design. Black and white objects are then filtered out and are not part of the embroidery stack.

Second, we filters for insulation color (blue). Green objects are duplicated and stored in two layers.

Third, we filter for circuit traces and sensors (purple). Blue objects include traces, hand-sketched sensors, and sensor stickers. Based on the marker tip, we reduce the thickness of traces in order to force embroidery software to treat them as outlines, as opposed to closed shapes that need to be filled. This step avoids stitching thick conductive traces, which wastes a lot of thread and makes them more visible in the final design.

Forth, we filter art markings (pink). We keep the thickness of these markings untouched.

The system ignores all others color. This features allows it to filter out thread colors during incremental design. Consequently, colors blue, purple, and pink should be preserved for sketching and not used as thread colors. 
%One limitation of this technique is that it doesn't allow users to color their designs during sketching or embroidery.

The system creates the embroidery stack: bottom layer contains purple objects (circuit traces and sensors), second layer contains blue objects (insulation), third layer contains pink objects (art pattern), top layer contains circuit bridges.

\subsection{Stitch Mapping and Embroidering}
Our system matches stitch types to embroidery stack as follows: triple running stitch (2 mm length) for circuit traces and sensors, zigzag stitch (2.0 mm width, 0.3 mm spacing) for insulation, satin stitch (2.0 mm width, 0.5 mm spacing) for art pattern, and single running stitch (9 mm length) for bridge stitch. Embroidery software converts the stack into layers of embroidery patterns. The system send these patterns to an embroidery machine via a wireless connection. 






Resnick’s Lifelong Kindergarten lab,
for example, has created programmable blocks that allow
children to build, explore, and program with materials that
can sense and act in the world in a contingent and
interactive fashion (Resnick & Silverman, 2005). These and other digital toolkits can substantially lower the barriers to engaging in physical computing.


A common distinction is between additive tools, like
3D printers and digital embroidery machines, which add
material to a substrate, and subtractive tools, like CNC
machines and laser cutters, which take material away.

Computer-controlled tools have a number of important
qualities. First, newcomers can produce objects with a
relatively high level of finish. Students often feel proud
when working with computer-controlled tools, as they can
make ‘‘real’’ products that look good (Blikstein, 2013a).
Second, compared to hand crafted objects, making multiple
identical or nearly identical items is easy and fast. An
analogy can be made to desktop printing compared with
hand drawing or typing a document. These multiples can be
identical, or they can be customized in systematic ways
(e.g., varying color, size, or material). Third, digital design
files are shareable with complete fidelity through computer
networks. Analog designs can also be shared, but doing so
is effortful for both the original designer and the person
hoping to reproduce the design.

accurate and reprducable 

Personal-scale
manufacturing tools
enable people that
have no special
training in
woodworking,
metalsmithing, or
embroidery to manufacture their own complex, one-of-a-kind artisan-style objects.


THIS COMPUTER-GUIDED
HOME SEWING MACHINE CAN
EMBROIDER ELABORATE
PATTERNS ACCORDING TO
DESIGNS FROM A SELECTION
OF ELECTRONIC BLUEPRINTS. 

Computer-guided
embroidery machines are as skilled as an expert needle worker and can fabricate
ornate designs involving several different colors of thread and intricate patterns.


e-textiles
embedding off-the-shelf miniature or thin-film-based electronic components like transducers etc. onto conventional dielectric fabrics as a motherboard or imparting electronic functions on the surface of fabrics by coating or printing or lamination

embroidered antenna: Embroidered Conductive Fibers on Polymer Composite for Conformal Antennas


http://onlinelibrary.wiley.com/doi/10.1002/adma.201400633/full
fabric-based sensors, many of which have been not only demonstrated as prototypes reported in papers but also widely used in real applications of wearable sensing and personal protection. Fiber-based sensors include strain sensors,[6b],[54, 140] pressure sensors,[6b],[,[21],142h],[209] chemical sensors,[141] as well as optical and humidity sensors[142]
Similarly, fiber-based pressure sensors deployed various transduction techniques including capacitive,[21],[209a],[209b] piezoresistive,[6b],[140h][148],[209c–e], piezoelectric[209f],[209g] and optical types.[209h] Capacitive fabric pressure sensors comprise embroidered electrodes from conductive yarns and spacer fabrics between them. With a pressure measuring range from 0 to 100 kPa, they can be used to detect muscle activity[209a] and sitting posture.[21] There are also fabric sensors by measuring the capacitance at the crossed points of warp and weft conductive yarns.[209b] Apart from the parasitic capacitance and cross-talk between sensor units, these capacitive sensors always require complex reading out circuitry. 

Two way insulation 

Using a standard two-thread system 

Towards Culture 3.0–performative space in the public library


Fully integrated embroidery process for smart textiles
The method of production only
involves the embroidery stitching technique, and has great
versatility in terms of sensor size, surface texture and
integrating materials
The embroidery technique has been proposed as a
technique with great potential and versatility for smart
textiles [16] and was used to produce soft electrodes for
neonatal intensive care EEG monitoring [17].


Trace stopping/starting points: to prevent excessive backstitching over traces, extraneous or redundant stitch lines were manually removed
 That is why we did not use a run stitch with X spacing in 2d sensors.


Trim locations: with closely-spaced traces, thread trimming creates a tangle of loose thread that can result in unnecessary bulk and the risk of electrical shorts

most conductive yarn is not as conductive as copper wires. Therefore, with most textile wires only low power applications can be realized \cite{linz2008embroidered}. However this is not a great issue for e-textiles, many applications in textiles are sensing applications [e.g. Fully Integrated EKG Shirt based on Embroidered Electrical
Interconnections with Conductive Yarn and Miniaturized Flexible Electronics, Textile Integrated
Contactless EMG Sensing for Stress Analysis] which are low power applications. 

Steel fibers (Bekintex) however do have a sufficiently high conductivity, but poor feel, instead metalized fibers (Shieldex, X-Static) or yarn spun from thin wires and non conductive fibers (Elektrisola) \cite{linz2008embroidered}.


To introduce the conductive threads into fabric three main types of textile wiring can be identified: weaving ["New Assembly Technologies for Textile Transponder Systems], knitting [Fibre-Meshed Transducers Based Real Time
Wearable Physiological Information Monitoring System, "A Wearable Health Care System Based on Knitted
Integrated Sensors] and embroidery ["E-broidery: Design and fabrication of textile-based computing]. The first two are producing the conductive structures during the fabric making. They are both limited in the freedom of routing of the conductive thread. Weaving though does have the advantage that it can create a two layer electrical design. Furthermore weaving is least demanding on the thread stability. This means it can handle all sorts of conductive yarn.
Knitting creates stretchable garments with high comfort which is why it is used in T-shirts and underwear. In knitted materials, the threads – and also conductive threads – take many loops rather than the shortest distance between two points. This causes a larger resistance of knitted wiring compared to woven or embroidered wiring.


Embroidery is applied onto the garment in a later manufacturing stage which can be an advantage as it makes tailoring much easier. Furthermore it can create any arbitrary layout in 2D. Unfortunately, it is quite demanding on the thread. This limits its use to metal-coated fibers at least as top thread. Special variants like soutage embroidery are very open to different threads or even cables [Lin05a]. However this technology is not compatible with the interconnection technology described below, therefore it will not be further discussed here.

Embroidery of conductive yarn can be applied on woven and knitted materials which makes it applicable in both fields.


There are different approaches to permanently interconnect electronics with conductive textile wiring. To name the basic ones these are: gluing, soldering, embroidery, crimping. [New Interconnection Technologies for the Integration of
Electronics on Textile Substrates] All of these do have their advantages and disadvantages. Our embroidered curcuit can be interconnected with all four of these technologies.

Our goal is to support users from the time they sketch an idea on fabric to component attachment.
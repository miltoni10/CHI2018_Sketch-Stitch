\section{Related Work}
This paper builds on research in e-textiles, physical sketching, and personal and interactive fabrication.

Two main efforts, attaching flexible conductive materials on fabric to create fabric circuits; and integrating off-the-shelf electronics into fabric. and interconnection and packaging of electronics in textiles.
The accessibility, durability of these techniques and integrating electronic modules, interconnections or connectors in a manner that preserves the fabric's textile character. reliability of electronics, textile conductors and connections in textile while washing, draping, stretching, wearing, etc. \cite{linz2008embroidered}

Early techniques for integrating electronics and textiles focused on hand-sewing conductive thread through perforations in PCBs. Linz et al. have successfully attached surface-mount components to textiles with non-conductive adhesive, using compression to displace the adhesive and create secure mechanical and electrical connection [Contacting electronics to fabric circuits with nonconductive
adhesive bonding]. Conductors can also be integrated into textiles by weaving or knitting [e-TAGs: e-Textile Attached Gadgets]. However, this often enforces an orthogonal, constrained trace layout. Further, stitched traces can be easily electrically isolated from traces on the opposite side of the fabric [E-Textiles in the Apparel Factory:
Leveraging Cut-And-Sew Technology Toward the Next
Generation of Smart Garments], enabling multi-layer circuits [Multi-Layer E-Textile Circuits].



%EXAMPLES
%  Polar’s breast strap heart rate monitor is to be mentioned. The usual rubber band around the body has now been replaced by a conductive textile ribbon. The second group includes products like the MET5 jackets by The North Face, an electrical heating for the wearer’s body, or the music playing jackets by InteractiveWear.

\section{Electronic Embroidery}

\cite{Berglund:2015:SCA:2802083.2808413} used reflow soldering technique to attach surface-mounted LEDs to a solderable conductive yarn (Liberator 40 Silver). 
%stitched in a lockstitch structure using the embroidery machine
%durable for low-intensity

\cite{linz2008embroidered} investigates embroidery as a means for interconnecting conductive yarn with electronics modules.
To halve the resistance of a conductive thread, they used it as top and as bobbin thread.
%heat (speed) degrades the conductivity of the silver coated threads

\cite{linz2005embroidering} deomstrated how a thin electronic substrate (flex substrate) can be attached to fabric the needle punches and  goes through metalized contact pad. Pre punched pad created loose stitches. Also packing of the module.
% the needle stitches through a metalized contact pad. Hereby the conductive top thread builds a loop over the conductive pad. The tighter this loop is the better is the contact.

\cite{eichinger2007using} (CAD PCB layout software) early work on using PCB layout software to create embroidred circuits. targetd for people familiar with PCB layout software such as Eagle.



%Considerable effort was spentmanuallydesigningcircuitsandfiguringout the proper scaling to create pads on fabric that match up with packages. We spent that time in designing the stickers. significant work required to use an embroidery CAD softwarepackagetocreatevalidandrepeatable circuits on the sewing machine. Add his to the into of conductive embroidery patterns that serve as contact areas and senosrs

capacitive sensing and variable resistors to attaching standard PCB packages.

\cite{brechet2017cost} embroidered antenna

\cite{zeagler2012textile} embroidered jog-wheel sensor

\cite{roh2014textile} embroidered touch sensors  


\cite{Gowrishankar:2013:PRE:2493988.2494341} repository of specific motifs
with different resistance values that can be easily
incorporated into e-embroidery projects and used instead of
normal resistors. 


 Hannah Perner-Wilson and Mika Satomi[5]
have taken such an approach and explored different craft
techniques such as knitting to make sensors from soft
materials that mimic the behaviors of regular sensors. 

\cite{buechley2005quilt} construction kit

\cite{linz2006fully} ?


\subsection{E-textile Fabrication Techniques}

Electronic textiles: A platform for pervasive computing


E-textile research investigates the integration of electronics and computational elements into clothes, furniture, toys and other fabric artifacts. 
%In HCI: Applications, fabrication techniques, and educational tools.
Fabrication techniques for fabric circuits are based on traditional textile methods: weaving [], kitting [], and stitching [] with conductive thread; drawing [], silk-screening [], and sputtering [] using conductive ink/polymer; and iron-on of conductive textile []. %; and couching thin wires with non-conductive thread []. 
Unlike the other techniques, weaving and knitting are non-additive (i.e., the circuit is created during fabric production as opposed to added to exiting fabric).  Each fabrication technique requires a different set of tools, materials, and crafting skills. The flexibility and robustness of the resulting circuits depend mainly on the characteristics of the conductive material. The conductive materials also dictate the methods that can be used to attach hardware components to the circuit. Stitching conductive thread is the de-facto fabrication technique for fabric circuits among hobbyists and makers []. Stitching does not require abundance of tools and materials nor high levels of skill. Conductive thread maintains the flexibility of the base fabric and endures movement, bending and flexing \cite{castano2014smart}. In addition, the availability of off-the-shelf sewable electronic kits adapted for needlework (e.g., The LilyPad Arduino \cite{4487082} and Adafruit's FLORA and GEMMA\footnote{https://www.adafruit.com/}),  simplifies the attachment of electrical components to the circuit.
%Alos https://www.bareconductive.com/shop/

%PaperPulse: An Integrated Approach to Fabricating Interactive Paper


%Midas: Fabricating Custom Capacitive Touch Sensors to Prototype Interactive Objects (users define electrical connections)

Buechley et al. \cite{Buechley2009} and Post et al. \cite{5387040} have investigated e-textile techniques. Buechley focused on techniques to attach electrical components to fabric. 

Post demonstrated e-broidery...

Others who used sewing/embroidery machine...

Using machines for embedding electronics in textiles: A Layered Fabric 3D Printer for Soft Interactive Object

To adapt for a wider audience, e.g., children, students, workshops participants, a number of e-textile toolkits have been developed.

However, these have limited expressive power.

\subsection{Personal and Interactive Fabrication}
Personal fabrication tools, such as milling machines, laser cutters, and 3D printers enable users to rapidly transform an idea into a physical artifact. Interactive fabrication systems replace bath processing with direct and physical interactions between the user, the workpiece, and the machine throughout the fabrication process. The benefits of these systems include a continuous representation of the object of interest, allowing users to establish a closer relation with materials to better understand their properties and nuances [], more agency and control over the final outcome [], and by simulating the traditional craft ways of directly interacting with the physical form using tools they open up a range of new creative possibilities [].

A number of interactive fabrication systems enable physical sketching on the workpiece. We categorize these systems based on where they provide feedback to the users: on the workpiece, on a display, and no feedback.


\textbf{On the workpiece sketching and feedback} 

* ModelCraft: Capturing Freehand Annotations and Edits on Physical 3D Models 

* CopyCAD: Remixing Physical Objects With Copy and Paste From the Real World (3-axis milling machine)

* ExoSkin: On-Body Fabrication (hybrid)

* ReForm: Integrating Physical and Digital Design through Bidirectional Fabrication 

* MARCut: Marker-based Laser Cutting for Personal Fabrication on Existing Objects 

* Makers' Marks: Physical Markup for Designing and Fabricating Functional Objects


\textbf{On the workpiece sketching and on display visual feedback} 

* Tactum: A Skin-Centric Approach to Digital Design and Fabrication 

* kidCAD: Digitally Remixing Toys Through Tangible Tools

* deForm: aninteractive malleable surfacefor capturing 2.5 D arbitrary objects, tools and touch

\textbf{On the workpiece sketching and no visual feedback} 

* Interactive construction: interactive fabrication of functional mechanical devices

\textbf{On screen sketching and on the workpiece visual feedback }

%VAL: Visually Augmented Laser cutting to enhance and support creativity (projecting work on workpiece before fabrication)

%Sketch It, Make It: Sketching Precise Drawings for Laser Cutting?




\subsection{Physical Sketching}
Designers and architect prefer drawing with pen and paper during design exploration phase as the their ideas are still abstract and ambiguous and CAD requires precision: [] %https://pdfs.semanticscholar.org/cb08/94518c519d93ebb71d4ad40097dac33893a1.pdf